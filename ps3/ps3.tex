\documentclass[12pt, a4paper,answers]{exam}
\usepackage{graphicx,fullpage,complexity}
\usepackage{amsmath,amsthm,amssymb}
\usepackage{xcolor}
\definecolor{iitpkdcolor}{cmyk}{0,0.42,1,0} %iitpkd logo color
\usepackage[normalem]{ulem}
\usepackage{import}

\renewcommand\ULthickness{2.0pt}   %%---> For changing thickness of underline
\setlength\ULdepth{1.3ex}%\maxdimen ---> For changing depth of underline
\newcommand{\N}{\mathbb{N}}
\newcommand{\HP}{\mathsf{HP}}
\newcommand{\ALL}{\mathsf{ALL}}
\newcommand{\TOTAL}{\mathsf{TOTAL}}
\renewcommand{\D}{\mathcal{D}}
\newcommand{\SD}{\mathcal{SD}}
%\newcommand{\BP}{mathsf{BP}}
%\newcommand{\}{def}
\begin{document}
	\newcommand{\imp}{$\Rightarrow$}
\newcommand{\ovl}{\overline}
\newcommand{\lem}{\le_m}
\newcommand{\coSD}{co\SD}
\newcommand{\doubleimp}{$\Leftrightarrow$}
\newcommand{\sd}{\SD}
\newcommand{\cosd}{co\SD}
\newcommand{\s}[1]{\Sigma_{#1}}
\newcommand{\pii}[1]{\Pi_{#1}}
\newcommand{\di}[1]{\Delta_{#1}}
\newcommand{\calp}{\mathcal{P}}
	%\thispagestyle{empty}
	\noindent
	\begin{minipage}[l]{0.1\textwidth}
		\noindent
		\includegraphics[width=2.4\textwidth]{iitpkd}
	\end{minipage}
	\hfill
	\begin{minipage}[c]{0.8\textwidth}
		\begin{center}
			{\large \textsc{\textcolor{iitpkdcolor}{Indian Institute of Technology Palakkad}} \par
				\small	Department of Computer Science and Engineering	\par
				\large	CS5616 Computational Complexity \par 
				\small January -- May 2024}
		\end{center}
	\end{minipage}
	\par
\vspace{2mm}
\hrule
\vspace{2mm}
\begin{minipage}{0.9\textwidth} 
	\textsf{Problem Set} -- 3  \hfill  \textsf{Total Points} -- 50
	
	{\small \textsf{Name}: Neeraj Krishna N   \hfill \small \textsf{Given on} 29 Feb}

	
	{\small \textsf{Roll no}: 112101033 \hfill \small \textsf{Due on} 11 Mar}
\end{minipage}
	\vspace{0.2in}
\noindent

\textbf{Instructions}
\begin{itemize}   \setlength\itemsep{0.1mm}
	\item {\sf  Use of resources other than class notes and references is forbidden.}
	\item {\sf Collaboration is not allowed. Credit will be given for attempts and partial answers.}
\end{itemize}


\begin{questions}

\question[10] (\textbf{RE vs co-RE}) Show that the set \[ \{M \mid M \text{ halts on all inputs of length less than $42$}\}\] is recursively enumerable, but is its complement is not.

\begin{solution}
	\import{./q1}{q1.tex}
\end{solution}

	\question[10] (\textbf{Alternate definition for $\Delta_i$}) Let $A$ be any language. Define $\D^A$ be the class of all languages $L$ such that $L$ is decidable in $A$. Similarly, $\SD^A$ be the class of all $L$ such that $L$ is semi-decidable in $A$ and $\co\SD^A $ be the class of all languages whose complement is in $\SD^A$.
	
	\begin{parts}
	\part[5] Show that $\D^A = \SD^A \cap \co\SD^A$.
	
	\part[5] For any $i \ge 1$, by definition, $\Delta_i = \Sigma_i \cap \Pi_i$. Show that  \[ \Delta_i = \{L \mid \text{ there exists } A \in \Sigma_{i-1} \text{ such that } L \text{ is decidable in A}\}. \]
	\end{parts}

\begin{solution}
	\begin{enumerate}
		% \item \includefrom{./q2}{q2a.tex}
		% \item \import{./q2}{q2a.tex}
		\item[(a)] \import*{./q2}{q2a.tex}
		\item[(b)] \import*{./q2}{q2b.tex}
	\end{enumerate}
\end{solution}

% Uncomment the following to answer
%\begin{solution}
%	Write your answer here.
%	\begin{description}
%		\item[(a)] Answer to first part
%		\item[(b)] Answer to second part
%	\end{description}
%\end{solution}	

\question[10] (\textbf{Closure properties of $\Sigma_n, \Pi_n$}). Fix any $i \ge 1$. Show that $\Sigma_i$ as well as $\Pi_i$ are closed under intersection and union.
	%%	 %Uncomment the following to answer
	%	\begin{solution}
		%		Write your answer here.
		%	\end{solution}
		
	
	\begin{solution}
		% 
\textbf{Claim 1} : $\s{i}$ closed under union \imp
$\pii{i}$ closed under intersection

\textbf{Claim 2} : $\s{i}$ closed under intersection \imp
$\pii{i}$ closed under union

Proof of Claim 1 : 


$\s{i}$ closed under union

\imp
$\forall L_1, L_2 \in \s{i}, L_1 \cup L_2 \in \s{i}$

\imp
$\forall L_1, L_2 \in \s{i}, \ovl{(\ovl{L_1} \cap \ovl{L_2})} \in \s{i}$, by De Morgan's Leftrightarrow

\imp
$\forall \ovl{L_1}, \ovl{L_2} \in \pii{i}, {(\ovl{L_1} \cap \ovl{L_2})} \in \pii{i}$

\imp
$\forall L_1', L_2' \in \pii{i}, {(L_1' \cap L_2')} \in \pii{i}$

\imp
$\pii{i}$ closed under intersection

Proof of Claim 2 : 


$\s{i}$ closed under intersection

\imp
$\forall L_1, L_2 \in \s{i}, L_1 \cap L_2 \in \s{i}$

\imp
$\forall L_1, L_2 \in \s{i}, \ovl{(\ovl{L_1} \cup \ovl{L_2})} \in \s{i}$, by De Morgan's Leftrightarrow

\imp
$\forall \ovl{L_1}, \ovl{L_2} \in \pii{i}, {(\ovl{L_1} \cup \ovl{L_2})} \in \pii{i}$

\imp
$\forall L_1', L_2' \in \pii{i}, {(L_1' \cup L_2')} \in \pii{i}$

\imp
$\pii{i}$ closed under union


Suffices to prove

$\s{i}$ is closed under intersection and union



% not yet completed




		\import*{./q3}{q3.tex}
	\end{solution}

	
\question [20] (\textbf{Rice's theorem}) Identify if the following are (0) properties of SD languages, (1) non-trivial properties and (2) non-monotone properties. If (1) / (2) is true, apply Rice's theorems suitably and give your conclusions. A direct use of diagonalisation or reductions does not fetch any credit.

\begin{parts}
	\part[4] given a Turing machine $M$, $L(M)$ is not regular ?
	\part[4] given a Turing machine $M$, does there exist a non-empty regular set $L'$ such that $L' \subseteq L(M)$ ?
	\part[4] given $M$, does $M$ represent a DFA that accepts some string with equal number of $0$s and $1$s ?
	\part[4] given a Turing machine $M$, is $L(M) \in \Pi_{42}$ ?
\end{parts}

% Uncomment the following to answer
%\begin{solution}
%	Write your answer here.
%	\begin{description}
	%		\item[(a)] Answer to first part
	%		\item[(b)] Answer to second part
	%	\end{description}
%\end{solution}	

\begin{solution}
	\begin{enumerate}
		\item[(a)] \import*{./q4}{q4a.tex}
		\item[(b)] \import*{./q4}{q4b.tex}
		\item[(c)] \import*{./q4}{q4c.tex}
		\item[(d)] \import*{./q4}{q4d.tex}
	\end{enumerate}
\end{solution}


\end{questions}

\end{document}
