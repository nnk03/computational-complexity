\documentclass[12pt, a4paper,answers]{exam}
\usepackage{graphicx,fullpage,complexity}
\usepackage{amsmath,amsthm,amssymb, amsfonts}
\usepackage{xcolor}
\definecolor{iitpkdcolor}{cmyk}{0,0.42,1,0} %iitpkd logo color
\usepackage[normalem]{ulem}
\usepackage{import}
\renewcommand\ULthickness{2.0pt}   %%---> For changing thickness of underline
\setlength\ULdepth{1.3ex}%\maxdimen ---> For changing depth of underline
\newcommand{\N}{\mathbb{N}}
\newcommand{\HP}{\mathsf{HP}}
\newcommand{\ALL}{\mathsf{ALL}}
\newcommand{\TOTAL}{\mathsf{TOTAL}}
\renewcommand{\D}{\mathcal{D}}
\newcommand{\SD}{\mathcal{SD}}
%\newcommand{\BP}{mathsf{BP}}
%\newcommand{\}{def}
\import{./}{new_commands}
\usepackage{hyperref}
\hypersetup{
    colorlinks=true,
    linkcolor=blue,
    urlcolor=red,
    pdftitle={112101033\_PS4\_CC}
}

\begin{document}
	%\thispagestyle{empty}
	\noindent
	\begin{minipage}[l]{0.1\textwidth}
		\noindent
		\includegraphics[width=2.4\textwidth]{iitpkd}
	\end{minipage}
	\hfill
	\begin{minipage}[c]{0.8\textwidth}
		\begin{center}
			{\large \textsc{\textcolor{iitpkdcolor}{Indian Institute of Technology Palakkad}} \par
				\small	Department of Computer Science and Engineering	\par
				\large	CS5616 Computational Complexity \par 
				\small January -- May 2024}
		\end{center}
	\end{minipage}
	\par
\vspace{2mm}
\hrule
\vspace{2mm}
\begin{minipage}{0.9\textwidth} 
	\textsf{Problem Set} -- 4  \hfill  \textsf{Total Points} -- 50
	
	{\small \textsf{Name}: Neeraj Krishna N    \hfill \small \textsf{Given on} 15 Mar}
	
	{\small \textsf{Roll no}: 112101033  \hfill \small \textsf{Due on} 5 Apr}
\end{minipage}
	\vspace{0.2in}
\noindent

\textbf{Instructions}
\begin{itemize}   \setlength\itemsep{0.1mm}
	\item {\sf  Use of resources other than class notes and references is forbidden.}
	\item {\sf Collaboration is not allowed. Credit will be given for attempts and partial answers.}
\end{itemize}


\begin{questions}

\question[10] [\textbf{Resource ?}]

Consider the following definitions where $M$ is a valid Turing machine encoding and $x \in \Sigma^*$. We use $\bot$ for value being undefined. Check if the follows are a resource using Blum's axioms. In case they don't, give a formal argument. \\ (\textit{Hint}: Decidability of the policing language !)
\begin{parts}
	\part[5] (Head turns) 
	
	$h(M,x) = \begin{cases} 
		\# \text{ turns that input head of $M$ makes on $x$ before halting} & \text{ $M$ halts on $x$} \\
		\bot & \text{ otherwise}
	\end{cases}$
	\part[5] (Count) Let $q_0$ be a state in the finite control of $M$.
	
	$c(M,x) = \begin{cases} 
		\# \text{ of times $M$ visits a state $q_0$ until it accepts or rejects} & \text{ $M$ halts on $x$} \\
		\bot & \text{ otherwise}
	\end{cases}$
	
\end{parts}
\begin{solution}
    \begin{description}
        \item[(a)] \import{./q1}{q1a.tex}
        \item[(b)] \import{./q1}{q1b.tex}
    \end{description}
\end{solution}

% Uncomment the following to answer
%\begin{solution}
%	Write your answer here.
%	\begin{description}
%		\item[(a)] Answer to first part
%		\item[(b)] Answer to second part
%	\end{description}
%\end{solution}	


\question[10] [\textbf{Oblivious Turing machines}]

A Turing machine $M$ is \textit{oblivious} if for every input $x \in \Sigma^*$ and every $i \in \N$, the location of $M's$ head at the $i^{th}$ step of execution is only a function of $|x|$ and $i$. In other words, the head movement does not depend on $x$. For a time constructible $t(n)$ show that if $L \in \DTIME(t(n))$, then there is a two tape oblivious Turing machine that decides $L$ in $O(t(n)^2)$ time.

% Uncomment the following to answer
%\begin{solution}
%	Write your answer here.
%	\begin{description}
%		\item[(a)] Answer to first part
%		\item[(b)] Answer to second part
%	\end{description}
%\end{solution}
\begin{solution}
        \import{./q2}{q2.tex}
\end{solution}


\question[15] [\textbf{More on crossing sequences}] 

In class, we saw $o(\log \log n)$ space restricted Turing machines must be regular. Here, we show similar results for time restricted machines.
\begin{parts}
	\part [5] Show that there exists a non-regular language that can be accepted by a one tape $O(n\log n)$ time Turing machine.
	\part [10] Show that any language accepted by a one tape deterministic Turing machine in time $o(n\log n)$ must be regular.
\end{parts}

% Uncomment the following to answer
%\begin{solution}
%	Write your answer here.
%	\begin{description}
%		\item[(a)] Answer to first part
%		\item[(b)] Answer to second part
%	\end{description}
%\end{solution}
\begin{solution}
    \begin{description}
        \item[(a)] \import{./q3}{q3a.tex}
        \item[(b)] \import{./q3}{q3b.tex}
    \end{description}
\end{solution}


\question[15]
[\textbf{Definition of space}] 

In this question, we explain why the definition of space complexity counts the cells that the machine scans and not just the ones that are written. Let $L = \{w \#w \mid w \in \{0,1\}^*\}$.
\begin{parts}
	\part[5] Show that $L \in \DSPACE(\log n)$.
	\part[10] Show that there exists a non-determinisitic Turing machine that writes only on cell to the right of left end marker symbol of its worktape and accepts $\overline{L}$. The machine may scan any number of worktape cells without writing on them.
\end{parts}

\begin{solution}
    \begin{description}
        \item[(a)] \import{./q4}{q4a.tex}
        \item[(b)] \import{./q4}{q4b.tex}
    \end{description}
\end{solution}
%\begin{solution}
%	Write your answer here.
%	\begin{description}
%		\item[(a)] Answer to first part
%		\item[(b)] Answer to second part
%	\end{description}
%\end{solution}


\end{questions}

\end{document}
