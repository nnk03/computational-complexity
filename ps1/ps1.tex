\documentclass[11pt, a4paper,answers]{exam}
\usepackage{import}
\usepackage{graphicx,fullpage,complexity}
\usepackage{amsmath,amsthm,amssymb,amsfonts}
\usepackage[ruled,vlined]{algorithm2e}
\usepackage{xcolor}
\definecolor{iitpkdcolor}{cmyk}{0,0.42,1,0} %iitpkd logo color
\usepackage[normalem]{ulem}
\renewcommand\ULthickness{2.0pt}   %%---> For changing thickness of underline
\setlength\ULdepth{1.3ex}%\maxdimen ---> For changing depth of underline
\newcommand{\N}{\mathbb{N}}
\newcommand{\HP}{\mathsf{HP}}
%\newcommand{\BP}{mathsf{BP}}
%\newcommand{\}{def}
\begin{document}
	\newcommand{\imp}{$\Rightarrow$}
\newcommand{\ovl}{\overline}
\newcommand{\lem}{\le_m}
\newcommand{\coSD}{co\SD}
\newcommand{\doubleimp}{$\Leftrightarrow$}
\newcommand{\sd}{\SD}
\newcommand{\cosd}{co\SD}
\newcommand{\s}[1]{\Sigma_{#1}}
\newcommand{\pii}[1]{\Pi_{#1}}
\newcommand{\di}[1]{\Delta_{#1}}
\newcommand{\calp}{\mathcal{P}}
	%\thispagestyle{empty}
	\noindent
	\begin{minipage}[l]{0.1\textwidth}
		\noindent
		% \includegraphics[width=2.4\textwidth]{iitpkd}
	\end{minipage}
	\hfill
	\begin{minipage}[c]{0.8\textwidth}
		\begin{center}
			{\large \textsc{\textcolor{iitpkdcolor}{Indian Institute of Technology Palakkad}} \par
				\small	Department of Computer Science and Engineering	\par
				\large	CS5616 Computational Complexity \par 
				\small January May 2024}
		\end{center}
	\end{minipage}
	\par
	\vspace{2mm}
	\hrule
	\vspace{2mm}
	\begin{minipage}{0.9\textwidth} 
		\textsf{Problem Set} -- 1  \hfill  \textsf{Total Points} -- 50
		
		{\small \textsf{Name}: Neeraj Krishna N \hfill \small \textsf{Given on} 03 Feb}
		
		{\small \textsf{Roll no}: 112101033 \hfill \small \textsf{Due on} 11 Feb}
	\end{minipage}
	\vspace{0.2in}
	\noindent
	
	\textbf{Instructions}
	\begin{itemize}   \setlength\itemsep{0.1mm}
		\item {\sf  Use of resources other than class notes and references is forbidden.}
		\item {\sf Collaboration is not allowed. Credit will be given for attempts and partial answers.}
	\end{itemize}

\begin{questions}
	

\question[15] [\textbf{More properties of R and RE}] We saw in class that recursive languages are closed under complement. We will see more properties below.
\begin{parts}
	\part [5] Show that recursively enumerable languages are closed under union and intersection.
	\part [5] For a language $L$, and $i \ge 1$, $L^i = \{a_1a_2\ldots a_i \mid a_1,\ldots,a_i \in L\}$. The Kleene closure of $L$, denoted by $L^*$, is defined as $ \{\epsilon\} \cup \bigcup_{i\ge 1} L^i $. Show that recursive and recursively enumerable languages are closed under Kleene closure.
	\part [5] For languages $L_1, L_2 \subseteq \Sigma^*$, define $L_1 / L_2 = \{x \in \Sigma^* \mid \exists y \in L_2 ~xy \in L_1\}$. Show that if $L_1$ and $L_2$ are recursively enumerable, then so is $L_1/L_2$.
\end{parts}

% Uncomment the following to answer
\begin{solution}
	Write your answer here.
	\begin{description}
		\item[(a)] \import{./}{q1a.tex}
		\item[(b)] \import{./}{q1b.tex}
		\item[(c)] \import{./}{q1c.tex}
	\end{description}
\end{solution}


\question[10]
[\textbf{Non-determinism}] The machines that we saw in class have a single valued transition function and hence are \textit{deterministic}. A nondeterministic Turing machine is a machine where the transition function can take multiple values.
\begin{parts}
	\part[5] Give a rigorous formal definition of a non-deterministic Turing machine, including a definition of configuration, next configuration relation and acceptance.
	\part[5] Argue that deterministic machines and non-deterministic machines are equivalent (in the sense that they can simulate each other).
	\begin{solution}
		% \import{./}{q2.tex}
		\begin{description}
			\item[(a)] \import{./}{q2a.tex}
			\item[(b)] \import{./}{q2b.tex}
		\end{description}

	\end{solution}
\end{parts}

% Uncomment the following to answer
%\begin{solution}
%	Write your answer here.
%	\begin{description}
%		\item[(a)] Answer to first part
%		\item[(b)] Answer to second part
%	\end{description}
%\end{solution}


\question[15] [\textbf{Undecidability}]
\begin{parts}
	\part [5]Consider the problem of deciding if two given Turing machines accept the same set. Formulate this as a language and show that it is undecidable.
	\part [5] Consider the problem of deciding given a Turing machine and a state, whether it enters the state on some input. Formulate this as a language and show that it is
 undecidable.
 	\part [5] Show that if Membership problem is undecidable, then Halting problem is undecidable. [\textit{Note: We independently know the undecidability of both the problems. Nevertheless, this question asks to prove the implication.}]
	\begin{solution}
		% \import{./}{q3.tex}
		\begin{description}
			\item[(a)] \import{./}{q3a.tex}
			\item[(b)] \import{./}{q3b.tex}
			\item[(c)] \import{./}{q3c.tex}
		\end{description}
	\end{solution}
\end{parts}

% Uncomment the following to answer
%\begin{solution}
%	Write your answer here.
%	\begin{description}
%		\item[(a)] Answer to first part
%		\item[(b)] Answer to second part
%	\end{description}
%\end{solution}


\question[10] [\textbf{Decidable or Undecidable ?}]
Let $|M|$ denote the length of the Turing machine description. Are the following problems decidable ? Justify.
\begin{parts}
	\part[5] Does a Turing machine $M$ takes at least $|M|$ steps on \textit{some} input ?
	\part[5] Does a Turing machine $M$ takes at least $|M|$ steps on \textit{all} inputs ?
	\begin{solution}
		% \import{./}{q4.tex}
		\begin{description}
			\item[(a)] \import{./}{q4a.tex}
			\item[(b)] \import{./}{q4b.tex} 
		\end{description}
		
	\end{solution}
\end{parts}

% Uncomment the following to answer
%\begin{solution}
%	Write your answer here.
%	\begin{description}
%		\item[(a)] Answer to first part
%		\item[(b)] Answer to second part
%	\end{description}
%\end{solution}
\end{questions}

\end{document}
