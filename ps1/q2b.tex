\begin{enumerate}
\item 
Let $M = (Q, \Sigma, \Gamma, \vdash, \sqcup, \delta, s, t, r)$ be a deterministic Turing Machine 

We want to construct an equivalent Turing Machine 

$N = (Q', \Sigma', \Gamma', \vdash, \sqcup, \Delta, s', t', r')$

Now let

$Q' = Q$

$\Sigma' = \Sigma$

$\Gamma' = \Gamma$

$\Delta(p, z) = \{\delta(p, z)\}$

$s' = s$

$t' = t$

$r' = r$

Now we have to prove $L(M) = L(N)$

Proof of $L(M) \subseteq L(N)$ :


$x \in L(M)$

\imp
There is set of steps for $M$ which leads to terminal accept state $t$

\imp
Since $\Delta$ has only one transition for every possible tuple $(p, z)$, $N$ will also take the same steps as $M$

\imp
$x \in L(N)$

Proof of $L(N) \subseteq L(M)$ : 

$x \in L(N)$

\imp
There is a set of steps for which $N$ reaches $t'$

\imp
Since $\Delta$ has only one transition for every possible tuple $(p, z)$, it behaves like a deterministic turing machine

\imp
$M$ also takes the exact same transitions for input $x$ and reaches the accept state $t$

\imp
$x \in L(M)$

Hence $L(M) = L(N)$ and we have proved that every deterministic Turing Machine can be simulated using a Non deterministic Turing Machine

\item 
Let $N = (Q, \Sigma, \Gamma, \vdash, \sqcup, \Delta, s, t, r)$ be a non deterministic Turing Machine 

We want to construct an equivalent deterministic Turing Machine 

$M = (Q', \Sigma', \Gamma', \vdash, \sqcup, \delta, s', t', r')$ such that it accepts the same language as that of $N$

Let 

$Q' = 2 ^ {Q}$

$\Sigma' = \Sigma$

$\Gamma' = \Gamma$

$s' = \{s\} \in Q'$

$t' = \{ t \} \in Q'$

$r' = \{ r \} \in Q'$

Now for $\delta$,

Let $q' = \{q_1, q_2, \dots, q_i\}$ for some $i \in \{0, 1, \dots, |Q|\}$, where each $q_i \in Q$ (if $i = 0$, then $q' = \phi$)

$\delta(q', z) = \bigcup_{q_i \in Q}{\Delta(q_i, z)}$

Now, we have to prove $L(N) = L(M)$

Proof of $L(N) \subseteq L(M)$ : 

$x \in L(N)$

\imp
There exists a non deterministic set of choices for the machine $N$ which leads to the terminal state $t'$

\imp
Since, $M$ is in essence, simulating every possible transitions of $N$, there exists a transition $\delta(\cdot) $ for which $(t, z) \in \delta(\cdot)$ for some $z$

\imp
$M$ accepts $x$

\imp
$x \in L(M)$


$x \in L(M)$

\imp
There exists a set of transitions at the end of which $(t, z) \in \delta(\cdot)$

\imp
Then, if we take the same set of non deterministic choices in the non deterministic turing machine, we will reach $t$ in $N$

\imp
$x \in L(N)$

Hence showed that $L(N) = L(M)$ and hence every non deterministic turing machine can be simulated by a deterministic turing machine

(
This construction works only because all of the following are finite : 
$Q, \Sigma, \Gamma$
)

\end{enumerate}