\begin{claim}\label{q4:L_to_L_pad}
    $L \in \NSPACE(s(n)) \Rightarrow L_{pad} \in \NL$
\end{claim}
\begin{proof}
    Since $L \in \NSPACE(s(n))$, there is a turing machine $N$ accepting $L$ using $s(n)$ space\\
    Below is a description of turing machine $N_p$ which accepts $L_{pad}$\\
    $N_p$ = \\
    ``\\
    On input $y = x\#0^{2^{s(n)} - n - 1}$
    \begin{enumerate}
        \item Run $N$ on $x$ (note that we don't have to copy the input $x$, we can use the input tape as input tape for $N$ and simulate $N$ in work tape, which takes $s(n)$ space)
        \item Accept if and only if $N$ accepts
    \end{enumerate}
    ``\\
    Input is $y = x\#0^{2^{s(n)} - n - 1}$ and $|y| = 2^{s(n)}$ and the machine $N_p$ uses $s(n)$ space which is $\log (2^{s(n)}$. Hence $L(N_p) \in NL$\\
    We need to prove $L(N_p) = L_{pad}$\\
    $y = x\#0^{2^{s(n)} - n - 1}\in L_{pad}$\\
    \dimp $x \in L$\\
    \dimp $N$ accepts $x$\\
    \dimp $N_p$ accepts $y$\\
    \dimp $y \in L(N_p)$

    Thus $L(N_p) = L_{pad}$ and hence $L_{pad} \in NL$
\end{proof}

\begin{claim}\label{q4:coNL_to_coNSPACE}
    $L_{pad} \in co\NL$ \imp $L \in co\NSPACE(s(n))$
\end{claim}
\begin{proof}
    instead of directly constructing $x_{pad}$ use a counter to denote how many zeroes, we have given the turing machine

    $L_{pad} \in co\NL$\\
    Let $N_c$ be the turing machine which accepts $\overline{L_{pad}}$. The below is a turing machine $N$ which accepts $L$\\
    $N$ = \\
    ``\\
        On input $x$,
        Reserve $s(n)$ space in work tape for maintaining a counter (possible using $s(n)$ space since it is space constructible) and initialize it with the binary representation of $2^{s(n) - n}$
        \begin{enumerate}
            \item Start simulating $N_c$, whenever it asks for input, Check the following cases
            \begin{enumerate}\label{q4:peculiar_input}
                \item If input head is still valid (i.e head points to $x_i$ for some $i$), use this as the symbol read by $N_c$ and make the head movement corresponding to the transition of $N_c$
                \item If the input head is invalid and if the counter value is ${2^{s(n)} - n}$, then return \# as symbol read and if head moves right, decrement counter by one and don't change if head moves left
                \item Else, return 0 as the symbol to $N_c$ and decrement counter by one if the corresponding head movement is towards right and increment counter by one if corresponding head movement is towards left
                \item If counter value is 0, then return blank symbol as input symbol for $N_c$
            \end{enumerate}
            \item Accept if and only if $N_c$ accepts
        \end{enumerate}
    ``\\

    Note that this machine uses only $O(s(n))$ space\\
    \begin{proof}
        [Correctness]
        The non trivial part is the steps in \ref{q4:peculiar_input}. We know that input to $N_c$ should be of the form $x\#0^{2^{s(n)} - n - 1}$, i.e we know that after and if we can keep track of how many zeroes $N_c$ have read after \#, we can easily identify whether the machine $N_c$ would have read $0$ or $x_i$ for some $i$, This is the reason we are maintaining a counter and once input head goes past $x$, we will start using the counter to determine whether to return $0$ or \# or blank symbol. Thus $N_c$ will always get the correct input\\

        $x \in L$\\
        \imp
        $x\#0^{2^{s(n)} - n - 1} \in L_{pad}$\\
        \imp
        $N_c$ rejects $x\#0^{2^{s(n)} - n - 1}$\\
        \imp
        $N$ rejects $x$\\
        \imp
        $x \notin L(N)$\\

        $x \notin L$\\
        \imp
        $x\#0^{2^{s(n)} - n - 1} \notin L_{pad}$\\
        \imp
        $N_c$ accepts $x\#0^{2^{s(n)} - n - 1}$\\
        \imp
        $N$ accepts $x$\\
        \imp
        $x \in L(N)$\\
        Thus $L(N) = \overline{L_{pad}}$  
    \end{proof}
    Hence $L(N) \in co\NSPACE(s(n))$
\end{proof}

We know that $\NL = co\NL$ and combining this with claims \ref{q4:L_to_L_pad} and \ref{q4:coNL_to_coNSPACE}, we get $\NSPACE(s(n)) = co\NSPACE(s(n))$ for any space constructible $s(n) \ge \log n$






