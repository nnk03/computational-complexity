\documentclass[11pt, a4paper,answers]{exam}
\usepackage{graphicx,fullpage,complexity}
\usepackage{amsmath,amsthm,amssymb}
\usepackage{xcolor}
\usepackage{import}
\definecolor{iitpkdcolor}{cmyk}{0,0.42,1,0} %iitpkd logo color
\usepackage[normalem]{ulem}
\renewcommand\ULthickness{2.0pt}   %%---> For changing thickness of underline
\setlength\ULdepth{1.3ex}%\maxdimen ---> For changing depth of underline
\import{./}{new_commands}
\newcommand{\N}{\mathbb{N}}
\newcommand{\HP}{\mathsf{HP}}
\newcommand{\TOTAL}{\mathsf{TOTAL}}
%\newcommand{\}{def}
\begin{document}
	%\thispagestyle{empty}
	\noindent
	\begin{minipage}[l]{0.1\textwidth}
		\noindent
		\includegraphics[width=2.4\textwidth]{iitpkd}
	\end{minipage}
	\hfill
	\begin{minipage}[c]{0.8\textwidth}
		\begin{center}
			{\large \textsc{\textcolor{iitpkdcolor}{Indian Institute of Technology Palakkad}} \par
				\small	Department of Computer Science and Engineering	\par
				\large	CS5616 Computational Complexity \par 
				\small January May 2024}
		\end{center}
	\end{minipage}
	\par
	\vspace{2mm}
	\hrule
	\vspace{2mm}
	\begin{minipage}{0.9\textwidth} 
		\textsf{Problem Set} -- 2  \hfill  \textsf{Total Points} -- 50
		
		{\small \textsf{Name}: Neeraj Krishna N \hfill \small \textsf{Given on} 09 Feb}
		
		{\small \textsf{Roll no}: 112101033 \hfill \small \textsf{Due on} 16 Feb}
	\end{minipage}
	\vspace{0.2in}
	\noindent
	
	\textbf{Instructions}
	\begin{itemize}   \setlength\itemsep{0.1mm}
		\item {\sf  Use of resources other than class notes and references is forbidden.}
		\item {\sf Collaboration is not allowed. Credit will be given for attempts and partial answers.}
	\end{itemize}

\begin{questions}
	
\question[10] [\textbf{Properties of $\le_m$}] Show that $\le_m$ relation is  reflexive and transitive  over languages in $\Sigma^*$. Is $\le_m$ symmetric ? Argue.

% Uncomment the following to answer
\begin{solution}
	% Write your answer here.
	\import{./q1}{q1}
\end{solution}

\question[20] [\textbf{Reduction by containment !}] Let $L_1, L_2 \subseteq \Sigma^*$ where $L_1 \subseteq L_2$. Consider the statements \textbf{(1)} ``$L_2 \le_m L_1$'' \textbf{(2)} ``$L_1 \le_m L_2$'' 
\begin{parts}
	\part[5] Give a pair of languages  where \textbf{(1)} is true. Provide appropriate justification if no such pair exists. 
	\part[5] Give a pair of languages  where \textbf{(2)} is true. Provide appropriate justification if no such pair exists. 
	\part[10] Prove or disprove the following statements (a), (b). 		
	\begin{enumerate} \item[(a)] ``for any $L_1, L_2$ with $L_1 \subseteq L_2$, \textbf{(1)} is true.''  
		\item[(b)]  ``for any $L_1, L_2$ with $L_1 \subseteq L_2$, \textbf{(2)} is true.'' \end{enumerate}
\end{parts}
\begin{solution}
	% Write your answer here.
	\begin{enumerate}
		\item[(a)] \import*{./q2}{q2a}
		\item[(b)] \import*{./q2}{q2b}
		\item[(c)] \import*{./q2}{q2c}
	\end{enumerate}
\end{solution}

% Uncomment the following to answer
%\begin{solution}
%	Write your answer here.
%	\begin{description}
%		\item[(a)] Answer to first part
%		\item[(b)] Answer to second part
%	\end{description}
%\end{solution}


\question[10] 	[\textbf{Proof of Rice's theorem 1}] In the proof of Rice's theorem for showing undecidability done in class,  we assumed that the non-trivial property $\mathcal{P}$ is false for $\emptyset$.  Explain how can this assumption be removed.  [\textit{Hint: Use $\overline{\HP}$ !}]

\begin{solution}
	% Write your answer here.
	\import*{./q3}{q3}
\end{solution}
% Uncomment the following to answer
%\begin{solution}
%	Write your answer here.
%\end{solution}




\question[10] [\textbf{Rice's theorem ?}] Define the language $\TOTAL = \{M \mid \text{$M$ halts on all inputs} \}$.
\begin{parts}
	\part Describe the language $\overline{\TOTAL}$ in a way similar to the language $\TOTAL$.
	\part Show that the languages $\TOTAL$ as well as $\overline{\TOTAL}$ are not semi-decidable. 
\end{parts}

\begin{solution}
	% Write your answer here.
	\begin{enumerate}
		\item[(a)] \import*{./q4}{q4a}
		\item[(b)] \import*{./q4}{q4b}
	\end{enumerate}
\end{solution}

% Uncomment the following to answer
%\begin{solution}
%	Write your answer here.
%	\begin{description}
%		\item[(a)] Answer to first part
%		\item[(b)] Answer to second part
%	\end{description}
%\end{solution}
\end{questions}

\end{document}
